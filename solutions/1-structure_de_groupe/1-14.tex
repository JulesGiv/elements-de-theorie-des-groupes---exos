Soit $x$ un élément de $\R^*_+$, on a $f(x)=x$, donc $f$ est surjective, vérifions que c'est un morphisme :

\begin{align*}
    \forall x,y \in \R,\ f(xy) &= |xy|, \\
    &= |x||y|, \\
    &= f(x)f(y).
\end{align*}

C'est donc un épimorphisme de groupe, déterminons son noyau :

\begin{align*}
    \ker f &= \left\{ x\in \R^*,\ f(x) = 1 \right\}, \\
           &= \left\{ x\in \R^*,\ |x| = 1 \right\}, \\
           &= \{-1, 1\}.
\end{align*}



Soit $x$ un élément de $\R^*_+$, on a $g(x)=x$, donc $g$ est surjective, vérifions que c'est un morphisme :

\begin{align*}
    \forall x,y \in \R,\ g(xy) &= |xy|, \\
    &= |x||y|, \\
    &= g(x)g(y).
\end{align*}

C'est donc un épimorphisme de groupe, déterminons son noyau :

\begin{align*}
    \ker g &= \left\{ x\in \R^*,\ f(x) = 1 \right\}, \\
           &= \left\{ x\in \R^*,\ |x| = 1 \right\}, \\
           &= \mathbb{U}.
\end{align*}