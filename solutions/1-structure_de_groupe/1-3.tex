Montrons que le symétrique à droite de tout élément $a$ de $G$ est aussi son symétrique à gauche.

\begin{align*}
    aa' = e &\Rightarrow a'(aa') = a', \\
    &\Rightarrow (a'a)a' = a'. 
\end{align*}

En multipliant des deux cotés par le symétrique à droite de $a'$, on obtient :

\begin{align*}
    a'a = e.
\end{align*}

Ainsi, le symétrique à droite de $a$ est aussi son symétrique à gauche.

Montrons que le neutre à droite de $G$ est aussi un neutre à gauche, et donc un neutre tout court.

\begin{align*}
    \forall a \in G,\ ea &= (aa')a, \\
    &= a(a'a), \\
    &= a.
\end{align*}

Ainsi, le neutre à droite de $G$ est aussi un neutre à gauche.

$(G, \cdot)$ est donc un groupe.

On vérifie que pour $(\Z, -)$, la loi n'est pas associative, mais que 0 est un neutre à droite (et non à gauche) et que tout élément est symétrisable.