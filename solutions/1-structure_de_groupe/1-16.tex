\begin{abc}
\item Un élément de H qui commute avec tout les élements de G 
commute aussi avec tout les élément de H, d'ou $Z(G) \cap H \subset Z(H)$. De plus,
l'intersection de sous-groupes est un sous-groupe, donc $Z(G) \cap H \leq Z(H)$.

\item Soit $y \in f(Z(G))$, il existe $x \in Z(G)$ tel que $y = f(x)$. $f$ étant surjective, 
pour tout $z \in G'$, il existe $w \in G$ tel que $z = f(w)$. On a donc :

\[yz = f(x)f(w) = f(xw) = f(wx) = f(w)f(x) = zy.\]

D'où $y \in Z(G')$, et comme $f(Z(G))$ est un sous-groupe de $G'$ inclus dans $Z(G')$, on a bien $f(Z(G)) \leq Z(G')$.
\end{abc}