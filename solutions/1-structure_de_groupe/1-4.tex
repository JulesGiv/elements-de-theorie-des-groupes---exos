Tout les éléments étant simplifiables à droite implique que les applications :

\[
\begin{array}[t]{rl}
\tau_g^y : G &\rightarrow G \\
x &\mapsto yx,
\end{array},\ 
\begin{array}[t]{rl}
\tau_d^y : G &\rightarrow G \\
x &\mapsto xy,
\end{array}
\]

Sont injectives. Le cardinal de G étant fini, ces translations sont bijectives.

Ainsi, pour $a$ et $b$ fixé, les équations $a=xb$ et $a=bx$ ont chacune une unique solution.

En particulier, pour chaque élément $a$ de $G$, il existe des uniques $e_d^a$ et $e_g^a$ tel que $a=e_d^a a$ et $a=ae_d^a$.

Vérifions qu'ils sont égaux :

\begin{align*}
    \forall a \in G,\ aa&=aa, \\
    a(e_g^aa) &= (ae_d^a)a, \\
    ae_g^aa &= ae_d^aa, \\
    ae_g^a &= ae_d^a \text{ (Simplifiation à droite)}, \\
    e_g^a &= e_d^a \text{ (Simplifiation à gauche)}.
\end{align*}

Vérifions maintenant que tout les éléments ont le même neutre : 

\begin{align*}
    \forall a, b \in G,\quad ab&=ab,\\
    (ae^a)b &= a(e^bb), \\
    ae^ab &= ae^bb, \\
    ae^a &= ae^b \text{ (Simplifiation à droite)}, \\
    e^a &= e^b \text{ (Simplifiation à gauche)}.
\end{align*}

Ainsi, dans $G$, il existe un unique élément neutre $e$.

Reste à montre que chaque élément $a$ admet un unique inverse $a^{-1}$.

On sait que les équations $e = ax$ et $e = xa$ ont une unique solution chacune, notées respectivement $a^*_g$ et $a^*_d$. Vérifions qu'il est le même des deux cotés, et est donc l'inverse de $a$.

\begin{align*}
    \forall a \in G,\quad a &= a, \\
    a (a^*_g a) &= (a a^*_d) a, \\
    a a^*_g a &= a a^*_d a, \\
    a^*_g &= a^*_d \text{ en simplifiant à droite et à gauche.}
\end{align*}

Chaque élément possède un unique inverse, et $G$ possède un élément neutre pour la loi associative $\cdot$. 
Ainsi, $(G,\cdot)$ est un groupe.