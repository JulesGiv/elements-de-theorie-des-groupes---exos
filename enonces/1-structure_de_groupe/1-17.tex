Soit $S$ une partie non vide d'un groupe $G$ ; on pose :

\[C_G(S) = \{g \in G;\ gx = xg,\ \forall x \in S\}.\]

\begin{abc}
\item Vérifier que $C_G(S)$ est un sous-groupe de $G$.

$C_G(S)$ est appelé le \emph{centralisateur} de $S$ dans $G$. Si $S=\{x\}$, on le note $C_G(x)$ et on l'appelle le \emph{centralisateur} de $x$ dans $G$.

\item $Z(G)$ étant le centre de $G$, démontrer la relaion : $\displaystyle \bigcup_{x\in G} C_G(x) = Z(G)$

\item Pour $x\in G$, posons $H = C_G(x)$; Vérifier que $x\in Z(H)$.
\end{abc}