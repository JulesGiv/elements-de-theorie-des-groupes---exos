
Soit G un groupe fini d'ordre $n>1$, d'élément unité $e$.

On ne suppose pas G abélien, mais on suppose que G vérifie la condition $(\mathcal{D})$ de l'exercice 18 si dessus.

On note D l'ensemble des entiers positifs diviseur de $n$. On remarque que D contient au moins 1 et $n$.

\begin{enumerate}[label=\alph*)]
    \item Pour tout $d \in $ D, on désigne par $\alpha(d)$ le nombre d'éléments de G, d'ordre $d$; on a $\alpha(d)\geq0$. Justifier l'égalité :
    \begin{equation}
        n = \sum_{d \in \text{D}} \alpha(d)
    \end{equation}
    \item $\varphi$ désignant la fonction d'Euler, démontrer que : $(d \in \text{D et } \alpha(d)\neq 0)\Rightarrow \alpha(d) = \varphi(d).$
    \item En considérant un groupe cyclique $C_n$, justifier l'égalité :
    \begin{equation}
        n = \sum_{d \in \text{D}} \varphi(d)
    \end{equation}
    \item Déduire des résultats ci-dessus que l'on a $\alpha(n)>0$; en conclure que G est cyclique.
\end{enumerate}