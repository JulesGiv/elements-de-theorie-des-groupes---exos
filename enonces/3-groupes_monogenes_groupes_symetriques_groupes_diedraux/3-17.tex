Soit $G$ un groupe monogène, on pose $G = \langle x \rangle$.

\begin{enumerate}[label=\alph*)]
    \item Montrer que, quel que soit $\alpha \in \text{Aut}(G)$, $\alpha(x)$ est un générateur de $G$.
    \item On suppose $G$ cyclique d'ordre $n>1$. Etant donné un entier $k$ premier avec $n$ tel que $1\leq k \leq n-1$, on considère l'application $\begin{array}{rl}
    \lambda : G \rightarrow G\\
 a \mapsto a^k
\end{array}$. Vérifier que $\lambda \in \text{ Aut } (G)$. Démontrer alors qu'il existe un isomorphisme de groupes de Aut$(G)$ sur le groupe multiplicatif $G_n$ des éléments inversibles de l'anneau $\cycle{n}$; en conclure que le groupe Aut$(G)$ est abélien et d'ordre $\varphi(n)$.
        \item On suppose G monogène infini; prouver que le groupe Aut(G) est cyclique d'ordre 2.
    \item  A l'aide des résultats précédents, trouver des exemples de groupes non isomorphes dont les groupes d'automorphismes sont isomorphes.
\end{enumerate}